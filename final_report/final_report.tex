\documentclass{article}
\usepackage{lipsum}
\usepackage[utf8]{inputenc}
\usepackage[smartEllipses]{markdown}

\title{The installation, configuration and use of QFieldCloud}
\author{Pettersson, E. Jussila, A. Chau, N}
\date{29.11.2021}

\begin{document}

\maketitle

\section{Introduction}
Short introduction

% Ngoc
\section{Installation of QFieldCloud}


\subsection{Docker and docker-compose}

\subsection{Installation on VirtualBox}

% Eemeli
\subsection{Installation on RedHat Enterprise Linux 8 Server}

\subsection{Difference of different instances}

% Anssi
\section{Components of QFieldCloud and their configuration}
Let's go through the contents of .env file and different yaml files

\subsection{Minio}

\subsection{Caddy}

\subsection{Databases}

\subsection{Other components}

% Eemeli
\section{Managing QFieldCloud}

\subsection{Management scripts}

\subsection{Admin view}

% Ngoc
\section{The installation and configuration of client software}

\subsection{QField Sync plugin and QField Dev mobile application}

% Needs a writer
\section{Working with a cloud project on QGIS}

\subsection{The synchronization modes}

\subsection{The project types}

\subsection{Downloading data}

\subsection{Preparing a project in QGIS}

\subsection{Sending a project to the cloud}

\subsection{Synchronization of a cloud project to QGIS}

\section{Working with a cloud project on the mobile app}

% Ngoc
\section{Results}

\subsection{The working parts}

\subsection{The non-working components}

% Eemeli
\section{Discussion and conclusions}

%Moving on from standard Docker

%Including an external \texttt{.md} file:

%\markdownInput{example.md}

\end{document}
